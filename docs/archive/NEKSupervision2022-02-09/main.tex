\documentclass{beamer}

% Math Fonts
\usepackage{amsmath,amsfonts,amssymb,amsthm,bm,dsfont} % Math Tweaks
\usepackage{booktabs}
\usepackage{multicol}
\usepackage{graphicx}
\usepackage{adjustbox} % resize big features in beamer
\usepackage{float}

% Babel sillabation, Fish and Chips folks
\usepackage[british]{babel}%

% Use natbib bibliography
\usepackage[semicolon,authoryear]{natbib}

%Information to be included in the title page:
\title{Identification Design with Non-Binary Treatments}
\subtitle{Some caveats I would like to discuss}
\author{Alessandro Pizzigolotto}
%\institute{Overleaf}
\date{
NEK Group Supervision\\
\today}

\begin{document}

\frame{\titlepage}

% --------------------------------------------------------------

\section{Introduction}

\begin{frame}\frametitle{Introduction}
    
    \begin{itemize}
        % Starting from the Christmas break, among other things
        \item I took a 'step back' in my populism paper to analyse all the building blocks
        \item Focus on the identification design to integrate with the flourishing DIDs and TWFE literature
        \item Identification design using non-binary treatment and multiple periods 
    \end{itemize}

    \vspace{0.2em}

    \begin{itemize}
        \item It would solve some headaches: more convincing identification, soothe some potential endogeneity issues
        \item People have starting using new estimators in the job market \citep[\textit{e.g}][]{bib:becker2021}
    \end{itemize}

    % I am going to show you a general setting that is in line with my identification design and build from there to show you three of the issues that puzzle me. Mostly I am focusing on the identification of the estimands in this setting, and then I will jump if I have time to covariates and pre-treatment trends tests 

\end{frame}

% --------------------------------------------------------------

\section{Identification Design}

\begin{frame}{Identification}\label{slide:identification}

    \begin{itemize}
        \item Let $t \in \{1, \dots,  T\}$, $g \in \{1, \dots, G\}$ and $i \in \{1, \dots, N\}$ placeholders respectively for any time period (year), group (county) and individual.
    \end{itemize}

    \begin{equation*}
        Y_{igt} = \alpha_g + \phi_t + \beta_{post} D_{gt} + \gamma X_{igt} + \varepsilon_{gt}\label{eq:twfe}
    \end{equation*}
    
    \begin{itemize}
    
        \item $Y_{igt}$ denotes the outcome of individual $i$ in group $g$ at period $t$
        \item $\alpha_g$ and $\phi_t$ are respectively group and period fixed effects
        \item $\beta_{post}$ denote the coefficient of $D_{gt}$ a continuous treatment at $g$ in period $t$: in this special case $D_{gt} = 0 \ \forall \ g, t < \tau$ and $D_{gt} > 0 \ \forall \ g, t \geq \tau$, where $\tau$ is a constant time period (2009)
        \item \textit{ignore} $X_{igt}$ covariates for now
    
    \end{itemize}
    
\end{frame}


% --------------------------------------------------------------

\begin{frame}{Outline}\label{slide:full-outline}
  \tableofcontents
\end{frame}

% --------------------------------------------------------------

\section{Issues}

\subsection{Forbidden Comparisons in Non-Binary Treatment}

\begin{frame}[allowframebreaks]{Issue 1: Forbidden Comparisons}

    \begin{itemize}
        \item Until recently, TWFE estimators have been considered equivalent to difference-in-differences under (conditional) parallel trends and no anticipation assumptions
        \item However, many researchers \citep{bib:callaway2021a,bib:dechaisemartin2020,bib:goodman-bacon2021} % (and there is literally a new paper coming out every week)
        have emphasized that the TWFE regressions may not identify a \textit{convex} combination of treatment effects due to 'forbidden extrapolations' \citep{bib:borusyak2021a}

        \begin{itemize}
            \item even without variation of treatment timing, forbidden comparisons arise from comparing groups with different treatment intensity at the same period
            \item fail to estimate the weighted sum of treatment effects of each group $g$ after treatment by weighting some of the groups negatively
        \end{itemize}

    \end{itemize}

    \framebreak
    
    \begin{itemize}
        \item Following \citet{bib:dechaisemartin2018}, focusing on non-binary treatments

        \item two-groups $g \in \{h(igh), l(low)\}$ and two-period setting $t \in \{pre, post\}$ for simplicity

        \begin{equation*}
            \hat{\beta}_{post} = \frac{Y_{h2} - Y_{h1} - (Y_{l2} - Y_{l1})}{D_{h2} - D_{h1} - (D_{l2} - D_{l1})}\tag{Wald-DID}\label{eq:walddid}
        \end{equation*}

    \end{itemize}

    \framebreak

    \begin{itemize}

        \item Assume that group $h$ goes from 0 to 2 in second period, whereas $l$ goes from 0 to 1, and potential outcomes are linear in the number of units and constant over time, but differ for groups: $Y_{gt} = T_{gt}(0) + \delta_{g}d$ (with $d$ placeholder for treatment)

        \item Denominator becomes 1
        \item Numerator (under parallel trends) becomes $2\delta_{h} - \delta_{l}$, weighted sum of treatment effects BUT group $l$ is weighted negatively subtracting its treatment effect out from $\hat{\beta}_{post}$

    \end{itemize}

    \begin{center}
        \textbf{Which heterogeneity-robust estimator is \\ correct to use in my case?}
    \end{center}

\end{frame}

% --------------------------------------------------------------

\begin{frame}{Issue 1: Forbidden Comparisons}{Potential Solution}
    
    % First I was thinking about an estimator ruling out dynamic effects (i.e. groups' current outcome only depends on its current treatment, no memory)
    % \citet{bib:dechaisemartin2022} propose an estimator robust to continuous treatments that compares the outcome evolution of movers and stayers from t-1 to t, but I have no stayers after $\tau$ so I should assume a bandwidth to determine quasi-stayers, and calculate what they call the weighted average of movers potential outcome slope.
    % I don't think this is a good strategy

    \begin{itemize}
        \item Ruling out dynamic effects (no memory) \citep[][with quasi-stayers]{bib:dechaisemartin2022}: IMO not a good idea

        \item Special case of estimators allowing for dynamic effects % groups' current outcome also depends on all past treatments

        \begin{itemize}
            \item non-binary (continuous) treatment
            \item special of staggered design where all groups assume a different treatment at the same period
        \end{itemize}

        \item \citet{bib:dechaisemartin2021} estimator (\texttt{did\_multiplegt} in Stata) allows for this estimation

        \begin{itemize}
            \item equivalent to estimate a weighted sum of DIDs between pairs of groups' treatment intensity
            \item interpretation of the parameter as some average of the effect produced by one unit increase of treatment 
        \end{itemize}

    \end{itemize}

\end{frame}

% --------------------------------------------------------------

\begin{frame}{Issue 1: Forbidden Comparisons}{Other Caveats in Estimation}
    
    \begin{itemize}
        \item \citet{bib:callaway2021} propose a similar estimator but distinguish between

        \begin{itemize}
            \item \textit{level effect}: treatment effect of dose $d$, difference between group's potential outcome under treatment $d$ and untreated potential outcome
            \item \textit{slope effect}: causal response to an incremental change in the dose $d$ \citep{bib:angrist1995} (!)
            \item in the binary DIDs they coincide
        \end{itemize}

        \item \textbf{\citet{bib:dechaisemartin2021} does not automatically infer the slope effect!}

    \end{itemize}

    \begin{itemize}
        \item Robustness to repeated cross-sectional data not clear

        \begin{itemize}
            \item I was following same individuals with age $\geq 16$ from pre-treatment wave ($t-3$) with attrition at the tails of the panel using sampling waves
            \item I want to use repeated cross-section of individuals with age $\geq 16$ to increase power
        \end{itemize}

    \end{itemize}

\end{frame}

% --------------------------------------------------------------

\subsection{Covariates Selection}

\begin{frame}{Issue 2: Covariates Selection}{What are the 'Good' Controls?}
    
    \begin{itemize}
        \item I have been controlling for:

        \begin{itemize}
            \item Time-varying second-order polynomial of age
            \item pre-treatment (time-invariant) individual and household characteristics
            \item pre-treatment (time-invariant) group-level (county) characteristics
        \end{itemize}
    \end{itemize}

    \begin{center}
        \textbf{is the correct way of doing it? Any suggestion?}
    \end{center}

\end{frame}

% --------------------------------------------------------------

\subsection{Pre-Treatment Tests under stronger assumptions}

\begin{frame}\frametitle{Issue 3: Pre-Treatment Tests under stronger assumptions}
    
    \begin{itemize}
        \item Both \citet{bib:dechaisemartin2021} and \citet{bib:callaway2021} use stronger parallel trends assumption (can be conditional):
    \end{itemize}
    
    \begin{equation*}
        \mathbb{E}\left[Y_{gt}(\bm{0}_t) - Y_{gt-1}(\bm{0}_{t-1})\right] = \mathbb{E}\left[Y_{g^{\prime}t}(\bm{0}_t) - Y_{g^{\prime}t-1}(\bm{0}_{t-1})\right], \ \forall \ g \neq g^{\prime}
    \end{equation*}

    \begin{itemize}

        \item cumbersome to test with standard event-study plot with $R_{gt} = t - \tau + 1$ time relative to treatment : $Y_{igt} = \alpha_g + \phi_t + \sum_{r \neq 0} \mathds{1}\left[R_{gt} = r \right] \beta_r + \eta_{gt}$

         \item \citet{bib:rambachan2021} propose a sensitivity analysis of pre-trends test

         \begin{itemize}
             \item a) estimate pre-treatment analogue to the counterfactual first post-treatment period b) impose an (arbitrary) magnitude of the post-treatment violations that needs to be reported in the event-study plot  
         \end{itemize}

    \end{itemize}

\end{frame}

\begin{frame}\frametitle{title}
    
     \begin{center}
        \textbf{\LARGE Other suggestions, comments and \\ references are super welcome!}
     \end{center}

\end{frame}

%----------------------------------------------------------------------------- %
%   REFERENCES                                                                 %
%----------------------------------------------------------------------------- %

\begin{frame}[allowframebreaks]
  \frametitle{References}
  \bibliographystyle{agsm}
  \bibliography{papers}
\end{frame}

\end{document}

